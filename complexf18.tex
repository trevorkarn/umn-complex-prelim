\documentclass{article}

\usepackage{amsfonts}
\usepackage{amsmath}
\usepackage{amsthm}
\usepackage[margin=1in]{geometry}
\usepackage{hyperref}

\DeclareMathOperator{\res}{res}

\title{\href{https://math.umn.edu/sites/math.umn.edu/files/exams/complexf18.pdf}{Fall 2018 Complex Analysis Preliminary Exam}}
\author{University of Minnesota}
\date{}
\begin{document}
\maketitle

Where possible, computations have been also done using SageMath code available on GitHub at \\ github.com/tekaysquared/prelims (feel free to make pull requests!)

\begin{enumerate}

	\item Tell the values of $i^i$.
	
	\begin{proof}
		Recall that $z^i$ is given by $e^{i \log z}$ on a suitably defined branch of the logarithm.
		As long as we choose a branch whose branch-cut is not along the positive imaginary axis, we see
		$\log i = i \pi/2 + 2\pi  i k$ since $e^{\pi i /2}e^{2\pi i k} = e^{i \pi/2} = i$
		
		Thus, for $k \in \mathbb{Z}$, $i^i$ takes on the values of
		\begin{align*}
			e^{i \log i } = e^{ i(  \pi i/2 + 2\pi i k)} = e^{- \pi/2 + 2\pi k}
		\end{align*}
		which each lie on the real axis! \emph{Incroyable!}
		
	\end{proof}	
	\setcounter{enumi}{1}
	
	\item Write the Laurent expansion of $f(z) = \frac{1}{z^4-1}$ centered at $0$ and convergent in $|z|>1$.
	
	\begin{proof}
		Factor out $z^{-4}$ to see that 
		\begin{align*}
			f(z) &= \frac{1}{z^4 (1 - {1}/{z^4})}\\
			&= \frac{1}{z^4} \sum_{n=0}^\infty \frac{1}{z^{4n}} \\
%			&= \frac{1}{z^4} \left ( \right )
		\end{align*}
		which converges for $|1/z^4| <1$ which is to say for $|z|>1$.
		So then $f$ has a Laurent expansion
		\[f(z) = \sum_{n = - \infty}^{\infty} a_n z^{n} \]
		where 
		\[ a_n = \begin{cases}
						  1 & n = -4k \text{ for nonzero positive integers }k\\
						  0 & \text{otherwise}
		\end{cases}\]
	\end{proof} 
	
	\item Let $f$ be an entire function so that $\Re f(z)$ is \textit{nonnegative} for all $z \in \mathbb{C}$. Show that $f$ is constant.
	
	\begin{proof}
		Picard's little theorem states that if $f: \mathbb{C}\rightarrow \mathbb{C}$ is entire and nonconstant
		then the image of $f$ misses at most one point. If $\Re f(z)$ is nonnegative for all $z$, then
		$z = -1, -2$ are two points which are not contained in the image, and so $f$ must be constant.
	\end{proof}
	
	\item Evaluate $\displaystyle \int_0^\infty \frac{x^{1/5}}{1+x^2} dx$
	
	\begin{proof}
		Let $R > 1$, and let $C_R^+$ be the open semicircle of radius $R$ in the upper half plane.
		Let $\gamma_R = C_R^+ \cup [-R,R]$. Let the orientation of $\gamma_R$ be counter clockwise.
		
		Let $f(z) = \frac{z^{1/5}}{1+z^2}$. We will compute the integral 
		\[ \int_{\gamma_R} \frac{z^{1/5}}{1+z^2} dx.\]
		First, note that $f$ has simple poles at $z = \pm i$.
		The only pole contained inside of the closed curve $\gamma_R$ is the pole at $+i$ (since we
		took $R>1$).
		Then by the residue theorem, we know that
		\[ \int_{\gamma_R} f(z) dz = 2\pi i \res_i f.\]
		Since there is a simple pole at $i$, we may compute 
		$\res_i f = \lim_{z \rightarrow i} (z-i) \frac{z^{1/5}}{(z+i)(z-i)}= \frac{i^{1/5}}{2i}.$
		Substituting this into the equation above, we obtain
		\[ \int_{\gamma_R} f(z) dz = \pi i^{1/5},\]
		independent of $R$.
		
		Now consider splitting $\gamma_R$ up into the integral along $C_R^+$ and $[-R,R]$.
		Then \[\int_{\gamma_R} f(z) dz = \int_{C_R^+} f(z) dz + \int_{[-R,R]} f(z) dz\]
		Since the length of $C_R^+$ is $\pi R$, 
		we can bound the magnitude of $\int_{C_R^+} f(z) dz$ using the estimation lemma
		(also called the $ML$ lemma), by
		\begin{align*}
		 \left | \int_{C_R^+} f(z) dz \right | &\leq \pi R \max_{z \in C_R^+} \left | \frac{z^{1/5}}{1+z^2} \right |\\
		 &= \pi R \max_{z \in C_R^+}  \frac{R^{1/5}}{|1+z^2|} 
		 \end{align*}
		 and viewing $|\cdot |$ as distance, we obtain $|1 + z^2| > |z^2|$ on the upper half plane.
		 This allows us to bound 
		 \[ \left | \int_{C_R^+} f(z) dz \right | < \pi R \max_{z \in C_R^+}  \frac{R^{1/5}}{|z^2|} = \pi R \max_{z \in C_R^+}  \frac{R^{1/5}}{R^2}=\frac{\pi}{R^{4/5}}. \]
		 Taking the limit $R \rightarrow \infty$, we thus see that the portion of the integral along $C_R^+$ vanishes.
		 
		 We now turn our attention to the portion of the integral along the real axis.
		 \begin{align*}
		 \int_{[-R,R]} f(z) dz &= \int_{-R}^0 f(z) dz + \int_0^R f(z) dz\\
		 &= \int_{R}^0 f(-z) dz + \int_0^R f(z) dz\\
		 &= \int_R^0 \frac{(-z)^{1/5}}{1+(-z)^2} dz + \int_0^R f(z) dz \\
		 &= -\int_0^R \frac{(-z)^{1/5}}{1+z^2} dz + \int_0^R f(z) dz\\
		 &= - (-1)^{1/5} \int_0^R f(z) + \int_0^R f(z)dz\\
		 &=\left ( - (-1)^{1/5} + 1 \right ) \int_0^R f(z) dz \\
		 &=\left ( - e^{i\pi/5} + 1 \right ) \int_0^R f(z) dz 
		 %
		 \end{align*}
		 so taking $R \rightarrow \infty$ we see that 
		 \[ \pi i^{1/5} = \left (  1- e^{i\pi/5}  \right ) \int_0^\infty f(z) dz\]
		 or 
		 \[ \frac{\pi e^{\pi i/10}}{1- e^{i\pi/5} } = \int_0^\infty f(z) dz\]
	\end{proof}
	
	\item Determine the radius of convergence of the power series for $\log z$ at $z_0 = -4 + 3i$.
	
	\begin{proof}
		Let $R$ denote the radius of convergence of the power series of $\log z$ centered at $z_0$.
		Now, note that there is no logarithm which takes a value at $0$, since $e^w = 0$ is never true for $w \in \mathbb{C}$.
		Thus, the power series expansion can converge for a disk of radius at most $|-4+3i - 0| = 5$, and so 
		\[ R \leq 5.\]
		
		On the other hand, it is a theorem that if $\Omega$ is a simply connected subset of 
		$\mathbb{C}$ which does not contain $0$ then there is a branch of the logarithm which is holomorphic on $\Omega$. Observe that the open disk $D_{5}(-4+3i) := \{ z \in \mathbb{C} : |-4+3i - z | < 5\}$ is simply connected and does not contain zero. Thus, there is a branch of the logarithm (call it $\log_{D_5}$) which is holomorphic on $D_5$. Since we have constructed a disk of radius 5 on which there is a holomorphic logarithm, we see that 
		\[ R \geq 5.\]
		Since we have bounded $R$ both above and below by 5, we see that $R=5$.		
	\end{proof}
	
	\item By a suitable change of coordinates, write $w^2 = z^4+1$ in the Weierstrass form $y^2=x^3+ax+b$.
	
%http://www-users.math.umn.edu/~garrett/m/complex/notes_2014-15/10_elliptic.pdf
	Hartshorne
	
	\item Try to define $w$ locally as a holomorphic function of $z$ by the relation $w^5 - 5zw + 1 = 0$. For 
	which $z$ does this fail to some extent.\footnote{Follow the procedure in section three of P.G.'s notes: \href{http://www-users.math.umn.edu/~garrett/m/complex/notes\_2014-15/ERNPRHH.pdf}{http://www-users.math.umn.edu/~garrett/m/complex/notes\_2014-15/ERNPRHH.pdf}, Example 3.0.1.}
	
	\begin{proof}
		The holomorphic inverse theorem states that if $F(z,w)$ is a polynomial and $F(z_0,w_0)=0$ and
		$\frac{\partial F}{\partial w} (z_0, w_0) \neq 0$, then there is a holomorphic expression for $w$ in 
		terms of $z$ in a sufficiently small radius of $z_0$.
		Thus, we want to find all the points where
		$F(z,w)$ and $\frac{\partial F}{\partial w}(z,w)$ are simultaneously $0$.
		
		We now consider $F(z,w) =: P(w), P^\prime(w) \in \mathbb{C}(z)[w]$. We want to find nontrivial
		points where they are simultaneously $0$.
		They are certainly simultaneously $0$ when their GCD is $0$, so we compute it.
		First,
		\[ P(w) = \frac{w}{5} P^\prime(w) - 4zw +1 \]
		and then
		\[ P^\prime(w) = q(w) (-4zw + 1) - \]
	\end{proof}
	
\end{enumerate}


\end{document}